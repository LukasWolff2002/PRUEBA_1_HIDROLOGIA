\part{Ayudantia 1}

Se define el balance hidrologico general como:

\begin{equation}
    \frac{ds}{dt} = X_{Entrada}-Y_{Salida}
\end{equation}

Lo cual se puede desglozar en 4 terminos:

\begin{equation}
    \Delta S_L + \Delta S_S + \Delta S_Z + \Delta S_N = P + Q_{SA} + Q_{ZA} - (E +ET +Int + Q_{SE} + Q_{ZE})
\end{equation}

Donde:

\begin{itemize}
    \item $\Delta S_L$: Cambio de almacenamiento superficial como lagos o embalses
    \item $\Delta S_S$: Cambio de almacenamiento de suelo no subterraneo
    \item $\Delta S_Z$: Cambio de almacenamiento de suelo subterraneo
    \item $\Delta S_N$: Cambio de almacenamiento de nieve y glaciares
    \item P: Precipitacion
    \item $Q_{SA}$: Caudal Superficial Afluente
    \item $Q_{ZA}$: Caudal Subterraneo Afluente
    \item E: Evaporacion
    \item ET = Evapotranspiracion
    \item Int: Infiltracion
    \item $Q_{SE}$: Caudal Superficial Efluente y caudal subterraneo efluente
    \item $Q_{ze}$: Caudal subterraneo efluente
\end{itemize}

\textbf{NOTA si hay una variacion de altura considerarla}

Donde se define como:

\begin{itemize}
    \item Efluente: Caudal que sale de un sistema
    \item Afluente: Caudal que entra a un sistema
\end{itemize}

Se define como \textbf{acumulacion}:

\begin{equation}
    \frac{dv}{dt} = \frac{A\Delta H}{dt}= Q_A + P - (Q_E + Q_{ZE} + E)
\end{equation}

Con eso se puede obtener el aumento o disminucion de altura de un embalse.